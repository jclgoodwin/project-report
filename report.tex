\documentclass[authoryearcitations]{UoYCSproject}
\usepackage{todonotes}

\author{Joshua Goodwin}
\title{Layout of arguments in the Artoo tool}
% \date{}
\supervisor{Fiona Polack}
\BEng

\wordcount{?}

% \includes{Appendices \ref{cha:usefulpackages}, \ref{cha:gotchas} and
  % \ref{cha:deptfac}}

% \excludes{\autoref{cha:quoteex}}

\abstract{}

% \dedication{To}

% \acknowledgements{
%   I would like to thank
% }

\begin{document}

\maketitle
% \listoffigures
% \listoftables
% \renewcommand*{\lstlistlistingname}{List of Listings}
% \lstlistoflistings

% \cleardoublepage

% \part{Preliminaries}
% \label{sec:start}
% \thispagestyle{empty}\cleardoublepage

%%%%%%%%%%%%%%%%%%%%%%%%%%%%%%%%%%%%%%%%%%%%%%%%%%%%%%%%%%%

\chapter{Introduction}

\section{Artoo}

Artoo (Argumentation Tool) is a web based tool for drawing diagrams that represent arguments
in the syntax of the Goal Structuring Notation.

\section{The Goal Structuring Notation}

The Goal Structuring Notation allows structured arguments to be drawn in the form of graphs.
Nodes within GSN arguments represent ``claims, strategies, evidence, and context/assumption/justification annotations'';
edges represent relationships between these.

%%%%%%%%%%%%%%%%%%%%%%%%%%%%%%%%%%%%%%%%%%%%%%%%%%%%%%%%%%%

\chapter{Literature review}

In 1736, Leonhard Euler solved the Seven Bridges of Ka\"{o}nigsberg problem by drawing a graph \citep{ismail2009some}.
\todo{Maybe cite Euler's actual paper?}
\todo{``\ldots and graph theory was born?''}

In 1962, William T. Tutte  published ``How to Draw a Graph'', which introduced the problem of graph drawing \ldots

\citet{huang2007effects} divide graphs into two groups: abstract graphs and domain graphs. GSN arguments fall into the latter category \ldots therefore \ldots

\todo{Some things go here. Once I've decided which ones are more and less relevant, I can add and remove detail as appopriate.}

\section{[section title]}


The Gestalt theories of visual perception.

\todo{Clearly, Gestalt psychology is about a lot more than just visual perception!}



One 

%%%%%%%%%%%%%%%%%%%%%%%%%%%%%%%%%%%%%%%%%%%%%%%%%%%%%%%%%%%

\bibliography{report}

\end{document}