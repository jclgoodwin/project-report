\documentclass[authoryearcitations]{UoYCSproject}
\usepackage{todonotes}

\author{Joshua Goodwin}
\title{Layout of arguments in the Artoo tool}
% \date{}
\supervisor{Fiona Polack}
\BEng

\wordcount{?}

% \includes{Appendices \ref{cha:usefulpackages}, \ref{cha:gotchas} and
  % \ref{cha:deptfac}}

% \excludes{\autoref{cha:quoteex}}

\abstract{}

% \dedication{To}

% \acknowledgements{
%   I would like to thank
% }

\begin{document}

\maketitle

% \listoffigures
% \listoftables
% \renewcommand*{\lstlistlistingname}{List of Listings}
% \lstlistoflistings

% \cleardoublepage

% \part{Preliminaries}
% \label{sec:start}
% \thispagestyle{empty}\cleardoublepage

%%%%%%%%%%%%%%%%%%%%%%%%%%%%%%%%%%%%%%%%%%%%%%%%%%%%%%%%%%%

\chapter{Introduction?}

\section{Artoo}

Artoo (Argumentation Tool) is a web based tool for drawing diagrams that represent arguments
in the syntax of the Goal Structuring Notation.

\ldots

\section{The Goal Structuring Notation}

The Goal Structuring Notation allows structured arguments to be drawn in the form of graphs.
Nodes within GSN arguments represent ``claims, strategies, evidence, and context/assumption/justification annotations'';
edges represent relationships between these.

GSN arguments are directed, multivariate \todo{?} graphs \ldots



\ldots



\section{Problem definition?}


\section{[what I'm going to do]}


This [report/project/thing] will \ldots

	\begin{enumerate}
		\item \begin{itemize}
			\item ,
		\end{itemize}
	\end{enumerate}



\chapter{Literature review}

In 1736, \citet{euler} solved the Seven Bridges of Ka\"{o}nigsberg problem by drawing a graph.
\citet{ismail2009some}
\todo{a strange paper to choose}
attribute the birth of graph theory to his use of this method.

In 1962, \citet{tutte} introduced the problem of graph drawing \ldots
He proved that  ...
Much of the more recent research has continued down this path

\ldots

\citet{huang2007effects} divide graphs into two groups: abstract graphs and domain graphs.
GSN arguments fall into the latter category \ldots  therefore \ldots
\todo{I don't think this is widely agreed upon}


\section{heuristics?}

Speed is one 

The Gestalt theories of visual perception, developed from a wider Gestalt theory of the mind  \ldots German psychologists \ldots the Berlin School \ldots inform 


\section{techniques?}


\subsection{}

\citet{Gansner:1993:TDD:630795.631037}
 




\chapter{}



\section{A force directed approach}

\ldots









\bibliography{report}

% appendices go here

\end{document}