\chapter{Method}

This chapter describes \ldots compares \ldots


\section{Requirements}

Informed by the findings of research discussed in chapter~\_, some requirements can be specified.



\section{Artoo}

Artoo consists of a general-purpose {\tt SVGGraph}  -- and a


\section{Experimental methodology}


\subsection{Test data}

\begin{itemize}
    \item Four graphs from \citet{aldenthesis}
\end{itemize}

Up to a point, GSN arguments are quite homogeneous. 


\section{Springy}

Springy.js is an existing JavaScript implementation of a force directed graph drawing algorithm.
It lays out graphs by representing nodes as point charges and edges as springs. Springy.js simulates the electrostatic forces of interaction between these point charges, and the extension of the springs caused by these forces, according to Coulomb's and Hooke's laws.


\subsection{Implementing }

Springy, as distributed [online], consists of a layout algorithm implemented in JavaScript (springy.js) along with a sample renderer for displaying a graph layout.

\section{Arbor}

Simple though the brute-force force directed algorithm implemented in Springy is to understand, it also inefficient.

The Barnes-Hut algorithm is $O$



\section{Dagre}

An key part of the layered layout approach is notion that directed graphs flow in a single direction (typically from top to bottom or from left to right). This is highly applicable to GSN arguments, particularly as specified by the GSN community standard document's guidelines \cite{gsnspec} (see )

Figure~\ref{fig:dagre1} shows

\begin{figure}
  \centering
  \includegraphics[width=\textwidth]{graphics/results/dagre.png}
  \caption{A na\"ive use of Dagre to render a graph from \cite{aldenthesis}}
  \label{fig:dagre1}
\end{figure}



\begin{figure}
  \centering
  \includegraphics[width=\textwidth]{graphics/results/dagre-enhanced.png}
  \caption{A less na\"ive use of Dagre compared to figure~\ref{fig:dagre}, with context, assumption and justification elements placed to the sides through the use of unrendered dummy elements}
  \label{fig:dagre2}
\end{figure}



\begin{landscape}

\section{Testing}



\subsection{Results}

\subsubsection{Dagre}

\begin{tabular}{ | c | c | c | c | }
    \hline
    Graph & Area & Running time & Edge crossings \\
    \hline
    1     & & & \\
    \hline
    2     & & & \\
    \hline
    3     & & & \\
    \hline
    4     & & & \\
    \hline
\end{tabular}



\end{landscape}
