\chapter{Implementation}

This chapter describes \ldots compares \ldots

\section{Springy}

\subsection{Hooke's law}

Hooke's law states that when a force is exerted on a spring, the distance by which it extends is proportional to that force:

$$
F = -kX
$$

(where $F$ is the force exerted on the spring, $k$ is a constant representing the spring's stiffness, and $X$ is the distance by which it extends)

\subsection{Coulomb's law}

``The magnitude of the electrostatic force of interaction between two point charges is directly proportional to the scalar multiplication of the magnitudes of charges and inversely proportional to the square of the distance between them.''

``The force is along the straight line joining them. If the two charges have the same sign, the electrostatic force between them is repulsive; if they have different sign, the force between them is attractive.'' \todo{reference}

In scalar form:

$$
|\mathbf F|=k_e{|q_1q_2|\over r^2}\qquad
$$

In vector form:

$$
\qquad\mathbf F_1=k_e\frac{q_1q_2}{{|\mathbf r_{21}|}^2} \mathbf{\hat{r}}_{21},\qquad
$$
