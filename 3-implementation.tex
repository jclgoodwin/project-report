\chapter{Implementation}

This chapter describes \ldots compares \ldots

\section{Springy}

Springy.js is an existing JavaScript implementation of a force directed graph drawing algorithm.
It lays out graphs by representing nodes as point charges and edges as springs. Springy.js simulates the electrostatic forces of interaction between these point charges, and the extension of the springs caused by these forces, according to Coulomb's and Hooke's laws.

\subsection{Hooke's law}

Hooke's law describes the relationship between the force exerted on a spring, and the distance by which it extends as a result of that force being exerted.
It states that the distance is proportional to the force:

$$
F = -kX
$$

(where $F$ is the force exerted on the spring, $X$ is the distance by which it extends, and $k$ is a constant representing the spring's stiffness)

\subsection{Coulomb's law}

Coulomb's law describes the electrostatic force of interaction between two point charges.

``is directly proportional to the scalar multiplication of the magnitudes of charges and inversely proportional to the square of the distance between them.''

``The force is along the straight line joining them. If the two charges have the same sign, the electrostatic force between them is repulsive; if they have different sign, the force between them is attractive.'' \todo{reference}

In scalar form:

$$
|\mathbf F|=k_e{|q_1q_2|\over r^2}\qquad
$$

(where $F$ F is the $q_1$ and $q_2$ are the two charges, $r$ is the distance between them, and $k_e$ is Coulomb's constant 

In vector form:

$$
\qquad\mathbf F_1=k_e\frac{q_1q_2}{{|\mathbf r_{21}|}^2} \mathbf{\hat{r}}_{21},\qquad
$$


Coulomb's law is [analogous to] Newton's law, which describes the gravitational force between two masses.
But, if it is assumed that nothing can have negative mass, the gravitational force is always attractive, whereas the electrostatic force described by Coulomb's law can be repulsive (if the charges of both particles have the same sign).

\subsection{Implementing }

Springy consist





\section{Arbor}

Simple 

