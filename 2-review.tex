\chapter{Research}


\section{History}

In 1736, \citet{euler} solved the Seven Bridges of Ka\"{o}nigsberg problem by drawing a graph.
\citet{ismail2009some}
\todo{a strange paper to choose. perhaps multiple citations?}
pinpoint his use of this method as the birth of graph theory.

Graph theory has many applications to modelling the real world \ldots

\subsection{Tutte-embeddings}

In 1963, \citet{tutte} popularised [says who] the problem of graph drawing when he presented 

["who showed that polyhedral graphs may be drawn in the plane with all faces convex by fixing the vertices of the outer face of a planar embedding of the graph into convex position, placing a spring-like attractive force on each edge, and letting the system settle into an equilibrium"]

This contained two important ideas: embeddings(, , and the application of an attractive spring-like attractive force \ldots 

\subsection{Knuth}

In the same year, \citet{Knuth63} described a system for drawing flowcharts that describe algorithms. \citet{battista} \todo{find out page number!} says this is [the first example of an algorithm for visualising information] although \citeauthor{Knuth63} cites earlier work done on a similar system by \citet{haibt1959}.

\citet{Knuth63} does not describe his flowcharts as graphs,

The annual symposium is 


\ldots

\citet{huang2007effects} divide graphs into two groups: abstract graphs and domain graphs.
GSN arguments fall into the latter category \ldots  therefore \ldots
\todo{I don't think this is widely agreed upon}







\section{What makes a good graph layout?}

GSN was intended to be a clearer way of presenting arguments than free text,
and the layout of arguments can affect this clarity.

The GSN arguments 

\subsection{The human gold-standard}

Real argument  can be found in literature. For example, 

The Gestalt principles of visual perception,
based on the wider Gestalt theory of the mind developed by German psychologists of the Berlin School in the late 19th and early 20th centuries,
are sometimes [?] used to [inform/justify?] graph layouts.

The principle 

Some studies have tried \ldots cognitive effects [Purchase et al?]

characterising 


\subsection{Edge crossings}

% There can be many practical motivations for minimising edge crossings.
% When P\`{a}l Tur\`{a}n worked in a brick factory during World War II,
% he wondered about the minimum number of crossings in a graph representing
% brick kilns, storage sites and the paths between them \ldots
% When a graph represents an electrical circuit, 

[But] more relevantly [which studies?] have found that edge crossings is the most important factor for comprehension

[who?] found that using the Gestalt principle of `closure' \ldots the viewer's mind instinctively joins up the lines




\section{Approaches to graph layout}

\subsection{Force directed algortihms}

Force directed graph layout algorithms are the descendents of \todo{as someone has observed?} \citet{tutte} \ldots a successor to \citet{tutte} spring embedding.

Force directed layout is simple to understand, being based on physical laws we encounter in the world 


[gansner 199*]

\subsection{grid}




\subsection{Layered graph drawing}

Layered or hierachical graph drawing  \ldots sometimes generalised as Sugiyama's method, after \ldots who pioineered this technique \ldots 




\section{[GSN-/Artoo-specific considerations?]}

[move to Requirements?]

The GSN specification \citep[section~2.2, pp.~26--27]{gsnstandard} gives some guidance on the layout of arguments \ldots



\subsection{Dangling edges}

suggests incomplete graph

\subsection{Directed cycles}

It can be useful to remove directed cycles from the internal representation of a graph
(before drawing them back in their correct, original directions)
-- for example, in order to assign a consistent rank to each node.
This is achieved by reversing certain edges.
\citet{gansner1993} show that a simple depth-first-search \ldots  Minimising the number of edges is more difficult, \citeauthor{gansner1993} \ldots

\subsection{Undirected cycles}

Undirected cycles can be elimiated by ignoring certain edges altogether, to produce a tree.  [citation needed]








\section{Implementation}

The Artoo tool is written mainly in JavaScript, numerous different 

Various [things] have been developed in response to percieved shortcomings in JavaScript \ldots

Programs written in the CoffeeScript\footnote{\url{http://coffeescript.org/}} language, designed to be more succinct and with some extra features, can be transcompiled to JavaScript \ldots this is interesting but \ldots

Brython \footnote{\url{http://www.brython.info/}} is a Python 3 interpreter written in JavaScript that can run in a web browser. \todo{performance overhead etc\ldots}

Haste\footnote{\url{http://haste-lang.org/}}, UHC-JS\footnote{\url{http://uu-computerscience.github.io/uhc-js/}} and GHCJS\footnote{} are compilers from Haskell to JavaScript; SMLtoJS\footnote{\url{http://www.smlserver.org/smltojs/}} is a Standard ML--to-JavaScript compiler. These are perhaps most interesting [?], since \citet{kennedyfuntrees} observed that a tree layout algorithm implemented in Standard ML ``reflects the structure of the abstract solution much better than an imperative language implementation''.



ASM\footnote{} is a strict subset of 

[However]


\section{Software development methodologies}

In \citeyear{67poorslop}, \citet*{67poorslop} explained ``Why Programming is a Good Medium for Expressing Poorly Understood and Sloppily Formulated Ideas''
(More recently, [sussman] gave a talk with the same title \ldots \todo{\url{http://vimeo.com/12060509}}.)

This idea will be 

\section{}

Reuse of existing software libraries is widely [?] understood to be [a very sensible idea]. There are , which \ldots for comparison with 

\begin{description}
  \item[Springy] is 
  \item[Arbor] is an implementation of the Barnes-Hut algorithm described in  
  \item[The JavaScript InfoVis Toolkit] contains algorithms ported from \ldots
  \item[vis.js]
\end{description}

The Graphviz library contains several algorithms, including \ldots [as described in \ldots]. Although it is written in C, Emscripten \footnote{\url{http://emscripten.org}} can compile LLVM bitcode (readily compiled from C code) [as described earlier?] to the ASM subset of JavaScript -

These implementations also provide lessons on the structure of \ldots



agile
