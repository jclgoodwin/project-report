\chapter{Research}

The problem of graph layout is just one branch of research in graph theory.
This chapter introduces the area, discusses different layout techniques, explores ways of evaluating the quality of graph layouts, and finally looks ahead to the task of extending the Artoo tool.


\section{Definitions}

\citet{huang2007effects} divide graphs into two groups: abstract graphs and domain graphs.
Graphical argumentation notations such as GSN appear to fall into the latter category, as do others such as those used fro modelling 

Graphs are relational structures, consisting of \emph{nodes} or \emph{vertices} connected by \emph{edges} or \emph{arcs}. In the domain of the GSN, \emph{element} may also be used for node, and \emph{connection} or \emph{relationship} for edge, although in general graph theory a graph's elements are both its nodes and edges.

A \emph{connected graph} is one where an unbroken path of edges exists between every pair of nodes. An $n$-connected graph is one which remains connected when any $n-1$ nodes are removed. 

\section{History}

In 1736, \citet{euler} solved the Seven Bridges of Ka\"{o}nigsberg problem by drawing a graph.
Some \cite{alexanderson2006cover} pinpoint his use of this method as the birth of graph theory as a branch of mathematics.

In 1963, \citet{tutte} popularised the problem of graph drawing, showing how any 3-connected
graph could be drawn on the plane with straight lines and no edge crossings.

In the same year, \citet{Knuth63} described a system for drawing flowcharts that describe algorithms. \citet{battista} suggests this was ``perhaps the first paper to present an algorithm for drawing a graph for visualisation purposes'', although \citeauthor{Knuth63} cites earlier work on a similar system by \citet{haibt1959}.



\section{Approaches to graph layout}

A range of different categories of graph layout algorithm have been developed over time,
varying in scope and efficiency from those suitable for general graphs to more efficient algorithms focused on particular categories of graph (binary trees, for example).
\citet{handbook} provides a recent broad and deep overview.
The categories of algorithm considered most relevant to the automatic layout of GSN arguments are summarised here.



\subsection{Force directed algorithms}

Force directed algorithms are relatively simple to understand, and target no particular type of graph.
\citet{handbook:forcedir} says this combination of simplicity and flexibility has made them particularly attractive,
spawning many variations and being widely implemented.

As \citet{handbook:forcedir} observes, \citet{tutte}'s algorithm for drawing a 3-connected graph with straight lines and no crossings, using a method using Barycentric coordinates, is considered the origin of force directed algorithms.

The ease of understanding associated with force directed methods is most true when ideas from the physical world are used. \citet{eades84} pretends a graph's nodes are steel rings, and its edges are springs connecting the rings; the nodes are moved to random positions, and the subsequent reaction of the mechanical system is simulated to produce a graph layout.

\citeauthor{eades84}'s springs are in fact \emph{spring-like} objects, responding not according to Hooke's law (an approximation of the actual physical responses of springs and elastic bodies) but rather exerting a force proportionate to the logarithm of their extension -- so that the force between far-apart nodes is less great.

Hooke's law states that the extension of a spring is proportional to the force exerted on it:

$$
F = -kX
$$

(where $F$ is the force exerted on the spring,
$X$ is the distance by which it extends,
and $k$ is a constant representing the spring's stiffness)

\citeauthor{eades84} uses the following formula in its place:
$$
F = c_1 \times \log(d \div c_2),
$$

(where $c_1$ and $c_2$ are constants

The steps of the algorithm are summarised as follows:

\begin{enumerate*}
\item Move nodes to random positions
\item Calculate the force on each node
\item Move the vertex $c_4 \times (force on vertex)$
\item Repeat steps 2--3, perhaps 100 times
\end{enumerate*}



\citet{SPE:SPE4380211102} extend the idea of forces between the nodes themselves, inspired by atomic particles and celestial bodies. Like \citeauthor{eades84} they are not afraid to play fast and loose with the real physical laws while still being inspired by them. 

In general, many different combinations of weak and strong, repulsive and attractive forces can be used to achieve different results.

Springy\footnote{\url{https://github.com/dhotson/springy} \label{fn:springy}} is a JavaScipt implementation of a force-directed layout  

\subsubsection{Manipulation}

The ``move all nodes to random positions'' step can be replaced by a less random event initiated by the user -- where a user moves one of the nodes, perhaps by clicking and dragging on a graphical display of the graph, and the system responds.

\subsubsection{Efficiency}

A clear drawback of force directed methods is their computational complexity, which becomes a problem for large graphs.
GSN arguments are typically relatively small -- for example, the largest in \_ is has \_ nodes -- but, nevertheless, [something].
Various ways have been found to improve their efficiency.

The Barnes-Hut algorithm is applicable to n-body simulations \citet{quigleyfade} 

Fruchterman and Reingold 

arbor.js\footnote{\url{http://arborjs.org/}} is a JavaScript implementation that uses the Barnes-Hut algorithm 

[Eades]

\citet{handbook:forcedir}

\subsubsection{Non-point nodes}

A deficiency of the methods discussed thus far is that they are concerned purely with abstract graphs, and [as a symptom of this] assume that all nodes are single points. As [[gasner]] points out, 


gansner 199*




\subsection{grid}




\subsection{Layered graph drawing}

Layered or hierarchical graph drawing techniques are sometimes generalised as \emph{Sugiyama's method}, after the work of \citet{4308636}, who presented a method for laying out directed graphs. Many of the steps directly address particular ``readability elements'', which are some of the aesthetics mentioned in section.

\begin{enumerate}
\item Order the nodes in a hierachy, based on the directions of the edges
\item Order the edges in such a way that minimises edge crossings. [Purchase] has shown empirically that this improves comprehension.
\item Decide [?] the horizontal positions of nodes
\item
\end{enumerate}

The steps form a rough framework to be filled in -- in this way, the method has formed a starting point for enhancements and variations just like the force directed idea, despite being less fundamentally simple.

Whereas force directed techniques typically produce layouts where 


Although 



\section{[GSN-/Artoo-specific considerations?]}


\section{}



\subsection{Dangling edges}

suggests incomplete graph



\subsection{Directed cycles}

It can be useful to remove directed cycles from the internal representation of a graph
(before drawing them back in their correct, original directions)
-- for example, in order to assign a consistent rank to each node.
This is achieved by reversing certain edges.
\citet{gansner1993} show that a simple depth-first-search \ldots  Minimising the number of edges is more difficult, \citeauthor{gansner1993} \ldots



\subsection{Undirected cycles}

Undirected cycles can be eliminated by ignoring certain edges altogether.  [citation needed]



\section{What makes a good graph layout?}

The GSN was intended to be a clearer way of presenting arguments than free text.
The process of breaking down arguments into their constituent parts achieves some of this clarity,
but the notation's graphical nature also appears to be important.
This raises the question of whether the particular layout of graphs -- in general, and specifically GSN arguments -- can affect their comprehensibility.

Some attempts have been made to enumerate the vital characteristics of a good graph layout, often in order to understand the trade-offs between different algorithms running times and the quality of the layouts produced.


\subsection{Quality metrics}

\citet{Himsolt95comparingand} compared a total of 11 different graph layout algorithms, ranging from force directed to more specialised ones only suitable for particular categories of graph (such as planar and directed acyclic).
As well as running time, and six quantitative layout-related criteria -- ranked in order of significance after observing the layouts produced -- there is included a ``personal rating'' (on a scale of 1--5), based on the judgements of colleagues upon viewing the layouts produced.
However, details of the experimental method used, and detailed statistical results, are not provided.
\todo{found that force directed algorithms produce good layouts for general graphs, but have the worst running time; DAG is also good with a better running time; others not relevant to this project}

\citet{DiBattista1997303} compared four algorithms, providing more detailed results.
Improving on \citeauthor{Himsolt95comparingand}'s use of about 100 graphs, and earlier work \todo{can't find the paper they mention (``S. Jones, P. Eades, A. Moran, N. Ward, G. Delott and R. Tamassia, A note on planar graph drawing algorithms, Technical Report 216, Department of Computer Science, University of Queensland (1991).'')} that used purely randomly generated graphs, they took 112 graphs from real-word applications and generated 11,582 variations in total. This is a good example to follow \ldots

They used implementations of the algorithms to lay out these graphs, and evaluated the resulting layouts according to nine quality metrics:

\begin{description}
    \item[Area]
``area of the smallest rectangle with horizontal and vertical sides covering the drawing''
    \item[Cross]
total number of edge crossings
    \item[TotalBends]
total number of edge bends
    \item[TotalEdgeLen]
total length of all edges
    \item[MaxEdgeBends]
``maximum number of bends on any edge''
    \item[MaxEdgeLen]
``maximum length of any edge''
    \item[UnifBends]
``standard deviation of the number of bends on the edges''
    \item[UnifLen]
``standard deviation of the edge length''
    \item[ScreenRatio]
``deviation from the optimal aspect ratio, computed as the difference between the width/height ratio of the best of the two possible orientations (portrait and landscape) of the drawing and the standard 4/3 ratio of a computer screen. ''
\end{description}

They boldly assert: ``It is widely accepted \ldots that small values of the above measures are related to the perceived aesthetic appeal and visual effectiveness of the drawing.''
However, being widely accepted does not always preclude being wrong.

The \textbf{ScreenRatio} metric is the least robust.
In general, the ideal aspect ratio will depend on various factors, such as where the graph is displayed.
Since the paper was published, data such as those published by Unity Technologies\footnote{\url{http://stats.unity3d.com/}} have shown that 4:3 is no longer the most common computer screen aspect ratio.
Standard paper sizes have a different aspect ratio still.
If aesthetic beauty is important, then perhaps the golden ratio \todo{reference} should be used instead.

However, being only one metric of nine, ScreenRatio has not been given undue significance, and it highlights a relevant point: linear layouts \todo{there should be a page in Di Battista's book}, for example, are likely to be difficult to fit into typical spaces.
The other metrics seem reasonable.


\subsection{Empirical evidence for the the validity of heuristics}

As graph theory, and in turn graph layout, has a broad range of applications, it follows that the reasons for giving significance to different aesthetic qualities can vary depending on the application.

For example, there can be many practical motivations for minimising edge crossings.
When \citet{JGT:JGT3190010105} worked in a brick factory during World War II,
he considered the minimum number of crossings in a graph representing
brick kilns, storage sites and the paths between them.
Where a graph represents an electrical circuit, edge crossings affect how the circuit can be printed on a circuit board.
Clearly, these applications are very different to laying out GSN arguments, and it is only a coincidence (albeit a useful one) that, for example, minimising edge crossings is considered important for both aesthetic appeal and solving practical problems.
More importantly, whereas it is objectively provable that meeting some criteria solves a problem in practical contexts, reasoning about optimising layout from a human perception point of view is more complicated and potentially subjective.

\citet{5674033} observe that ``Many graph layout algorithms that have been devised over
several decades have typically been designed in accordance with the intuitions of the algorithm designers.''
\citet{eades84}, for example, intuited that ``edge lengths ought to be about the same and the layout should display as much symmetry as possible''.
\citeauthor{5674033}'s observation highlights the need for these intuitions to be validated properly.

First, particular aesthetics have been evaluated. \citet{Purchase1997basis} evaluated three -- maximising symmetry, minimising edge crossings, and minimising edge bends -- which had been mentioned in literature describing desirable properties of drawings produced by various algorithms. \todo{go in to more detail?}
Nine drawings (fulfilling these aesthetics to varying degrees) of one of two graphs were shown to 84 participants,
who were asked questions requiring them to read the drawings.
Drawings with edge crossings and bends were found to correlate with errors in a statistically significant way, but symmetry -- a complexly defined metric -- appeared to have no significant effect.

Then \citet{Purchase1997which} reevaluated these aesthetics along with two more. ``The results show that there is strong
evidence to support minimising crosses, and weaker evidence for minimising the number of bends and maximising perceptual symmetry.''
Neither of the two additional criteria -- orthogonal structure, and the maximum angles between edges leaving nodes -- appeared to have much effect.

The studies show a useful, repeatable experimental method which \citet{PURCHASE1998647} has later adapted for the evaluation of algorithms, showing the trade-offs between different combinations of layout characteristics.
However, as \citet{Purchase1997which} acknowledges, the studies only investigate ``relational'' reading of drawings of abstract graphs, as opposed to ``interpretative'' reading which would be more relevant to the layout of GSN arguments.

One attempt to fill this gap has been \citet{storrle} showing that, particularly for inexperienced users, various criteria commonly recommended for the ``good'' layout of UML diagrams (diagrams in the Unified Modelling Language, used to model the structure of software systems) do improve comprehension. However, he has not looked at the effects of specific aesthetics.

In another paper, \citet{Purchase:2002:EEA:594512.594527} describe three different studies. Two of these measure users' performance with abstract graphs, much like previous studies \cite{Purchase1997basis, Purchase1997which, PURCHASE1998647}, but the third investigates an study of participants' preferences about UML diagram layout.

Both class diagrams and collaboration diagrams appear in the UML study; as the latter describe behaviour, compared to class diagrams which emphasise structure, they can be considered slightly closer to GSN arguments in terms of showing the steps of a process -- although the hierarchical, inheritance-showing nature of the class diagrams may in fact have more in common with GSN arguments. The method -- asking users about their preferences -- is different to that of other studies, and may be less reliable -- just as users claim that using keyboard shortcuts feels quicker, but are actually faster at using a mouse \citep[pp.~26]{tognazzini1992tog}, they may claim to prefer graph layouts which in fact are more error-prone -- but the reduced focus on specific tasks, along with the use of different types of UML diagram, perhaps makes its findings more broadly applicable to different contexts such as GSN argument layout. In the study, the importance of minimising edge crossings is once again confirmed, and the importance of orthogonality (aligning nodes and edges to a grid), and showing a clear direction of information flow (perhaps by having all directed edges point in the same direction), are identified. Another discovery is users' preference for combined adjacent inheritance edges, as shown in figure~\ref{fig:connectededges} -- this is not conventional in GSN diagrams, as discussed in section~\ref{section:humangold}, but could be adopted. Some of the conclusions are not clear: the study identifies possible confounding factors, such as that users' apparent preference for orthogonal layout -- contradicting the performance-based study described earlier in the paper, which found orthogonality to have no significant effect -- may in fact really be a manifestation of preferring straighter edges, as the orthogonal drawings had fewer bends.

\begin{figure}
  \centering
  \includegraphics[width=0.5\textwidth]{graphics/connected-inheritance-edges.png}
  \caption{Connected inheritance edges in a UML diagram \cite{kennysun}}
  \label{fig:connectededges}
\end{figure}


\citet{huang2007effects} looked at the effects of drawing conventions and edge crossings in sociograms -- domain graphs representing the relationships between people, organisations and other social entities -- using 27 participants who were asked to analyse twelve sociogram drawings. (As with UML diagrams, the relevance of sociogram layout to GSN argument layout may be questionable.) For each of six drawing conventions, a version was produced with many edge crossings and another produced with minimal crossings. As well as confirming the negative effect of edge crossings, they found the even distribution of nodes -- sometimes touted as a featured of aesthetically beautiful graphs, for example by \citet{SPE:SPE4380211102} -- to be a pitfall, recommending instead that ''the distances between nodes should reflect their relationships''. They offer further, more specific advice for optimising layouts for different sociogram-related tasks (group tasks and importance tasks) -- tasks which are likely to be different to those in which GSN arguments might feature.


Most recently, \citet{5674033} invited users to draw graphs :
compared ``formal'' (like the Artoo tool) and ``sketch'' interface modes.
One of

\subsection{Theories of perception \label{section:gestalt}}

The Gestalt principles of visual perception,
based on the wider Gestalt theory of the mind developed by German psychologists of the Berlin School in the late 19th and early 20th centuries,
are often \citep[pp.~136]{storrle} invoked in discussions of graph layout.
\citet{kennysun} evaluated the layouts of UML class diagrams produced by two commercial modelling tools, using common principles taken from various theories of perception.
In the absence any existing work specific to argumentation notations, it provides a good summary of the theories and attempts to apply them to the layout of graph structures.

Some of the principles summarised by \citet{kennysun} can be applied to GSN argument layout, as follows:

\begin{description}
    \item[Connectedness] Objects that are physically connected are perceived to be connected. This is extremely obvious, and is shown in almost every conventional representation of graphs and GSN arguments, but does importantly provide a formal motivation for the idea that nodes should be spaced sufficiently far apart that non-connected nodes do not overlap (see the problem of non-point nodes in force-directed graph layout). 

    \item[Similarity] Objects that have the same shape or colour, for example, appear grouped together.
    The GSN effectively exploits this law, as elements of each type have a common shape. As with the principle of connectedness, layout can support this by positioning nodes sufficiently far apart, so that nodes' shapes are clearly visible.
    
    \item[Proximity] ``Elements that are close to each other are grouped together.'' This is supported by \citet{huang2007effects}'s evidence-based recommendation that 
    
    \item[Familiarity] ``Elements are more likely to be grouped together if the groups seem familiar or meaningful.'' This principle could be 
    
\end{description}



\subsection{The human gold-standard \label{section:humangold}}

Similarly to the method of \citet{5674033}, One understanding of the ideal layout of a GSN argument can be reached by observing the layouts of arguments found ``in the wild'' -- like the method, , understanding 
Although it is not always clear whether these have been laid out ``manually'' or automatically, it is reasonable to assume that , 
Plenty of arguments can be found in literature, for example in \cite{Habli:2006:PPC:1183088.1183090} and  \cite{insilico}.

Layouts typically follow the guidelines described in the GSN specification \citep[section~2.2, pp.~26--27]{gsnstandard}, \todo{should I consistently call it a ''community standard [document]'' or something else?}
to ``enable the reader to perceive the logical flow of the argument being presented, and to enhance its readability.''
The specification suggests that arguments should flow down the page, starting with an abstract parent goal at the top, progressing downwards as this is refined into more concrete child strategies, goals and solutions -- this allows the hierarchy of the graph to be understood without looking at the directions of the connecting arrows.
Context, assumption and justification elements are placed to the sides, a positioning which reflects their semantic role.

Although the specification document does not fully justify its recommendations, they seem reasonable, but there there are instances of authors contravening them.

Figure~\ref{fig:aldencentral} shows \citet{royal} placing child strategies around a central claim, contrary to the top-down approach recommended. Two other arguments in the paper \cite[pp.~8--9]{royal} have a similar layout.
A rough visual observation highlights that -- compared to a theoretical layout following the official guidance to the letter -- it takes slightly longer to identify the top-level claim node, but offers these advantages:

\begin{itemize*}
\item Smaller area
\item Better aspect ratio
\item 
\end{itemize*}

The refinements of the strategies do mostly follow the standard guidance, continuing in a top-down (or bottom-up) order, and with context, assumptions and justifications placed at the sides. However, the positions of the children of strategy 1.1.1.4 (in the bottom right corner) are clearly incorrect, in a way that appears to have no advantages.

\begin{figure}
    \includegraphics[width=\textwidth]{graphics/aldencentral.jpg}
    \caption{An argument from \cite{royal}, demonstrating a liberal interpretation of the GSN standard layout guidance}
    \label{fig:aldencentral}
\end{figure}

Other literature  . As the authors of \cite{royal}'s authors include the original developer of Artoo, 

In \cite{gsnstandard}, layout guidance is not demonstrated using specific diagrams, but there are many examples elsewhere in the document. These follow the guidelines, but are often imperfect in other ways:

\begin{itemize*}
    \item As noted in section~\ref{sec:cramped}, the layout shown in figure~\ref{fig:crampedex1} is very dense; semantically related nodes are not positioned closer together, indicating that the author has not considered the principle of proximity to aid readers' understanding (see section~\ref{sec:gestalt}).
    \item Another example, shown in figure~\ref{fig:unalignedsiblings} \ldots \ldots
\end{itemize*}

This supports a conclusion that, although examples found in literature are not always are perfectly laid out, the imperfections are at least easy to identify.

\begin{figure}
    \includegraphics[width=\textwidth]{graphics/unaligned_siblings.pdf}
    \caption{A fragment of a GSN argument,
            from the GSN specification \citep[figure~42, section~2.3.6.5, pp.~34]{gsnstandard}}
    \label{fig:unalignedsiblings}
\end{figure}








\section{Implementation}

The Artoo tool is written mainly in JavaScript 

Various tools have been developed in response to perceived shortcomings in JavaScript, which should be considered before., 

Programs written in the CoffeeScript\footnote{\url{http://coffeescript.org/}} language, designed to be more succinct and with some extra features, can be transcompiled to JavaScript \ldots this is interesting but \ldots

Brython \footnote{\url{http://www.brython.info/}} is a Python 3 interpreter written in JavaScript that can run in a web browser. \todo{performance overhead etc\ldots}

Haste\footnote{\url{http://haste-lang.org/}}, UHC-JS\footnote{\url{http://uu-computerscience.github.io/uhc-js/}}
and GHCJS\footnote{} are compilers from Haskell to JavaScript;
SMLtoJS\footnote{\url{http://www.smlserver.org/smltojs/}} is a Standard ML--to-JavaScript compiler.
These are perhaps most interesting [?], since \citet{kennedyfuntrees} observed that a tree layout algorithm implemented in Standard ML ``reflects the structure of the abstract solution much better than an imperative language implementation''.

ASM\footnote{} is a strict subset of 

Because the, and in order to take advantage



